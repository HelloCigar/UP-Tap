%   Filename    : chapter_2.tex
\chapter{Review of Related Literature}
\label{sec:relatedlit}

\section{Importance of Attendance Tracking}

Attendance has become increasingly important in every organization, institution, and workplace to ensure accountability, productivity, and engagement. For example, in schools, it ensures that students are present, participating, and fulfilling their responsibilities. Taking students' attendance is important for monitoring their performance in class. Good attendance is usually linked to good class performance, and vice versa (Zhi, Ibrahim \& Aris, 2014).

\subsection{Traditional Attendance Methods}

The traditional method of taking attendance is through a manual roll call. According to Uniyal (2022), using manual attendance is cost-effective, simple to use, and remains functional during power interruptions. However, despite these advantages, manual attendance has several flaws such as time consuming like for the roll call method, according to (Mahato \& Suman, 2013, p. 5875). An average of 5 - 15 minutes is wasted for manual roll calls which is a lot of time that will be consumed during class or work time. Another one is that there is no integrity when the ledger sheets are the method of taking attendance as there is a possibility to fake another student’s attendance through forging another student’s name and signature plus it is also easy for the student to replace and erase someone already there.

\subsection{Biometric-Based Attendance Systems}

The Biometrics - fingerprint filled some of the gaps in manual attendance. According to (Walia \& Jain 2016), replacing the traditional way of taking an attendance to biometric fingerprint is a must as it fills the gaps in taking the manual attendance such as the roll call and paper based. The unique fingerprint of each person is a great idea to include in the field of attendance management. Even though a biometrics fingerprint attendance system is an ideal way to have validity, reliability, etc., there are still possible problems that may occur if we totally applied this way alone itself. According to (Truein, 2024), there is a possibility to have an issue in terms of the target’s biometric recognition when the part of their finger they use to register to identify their fingerprint is wounded or injured as the current sensors are not capable to detect deeply within the wound plus dirty and dusty fingerprint may give the sensor a difficulty to analyze the person’s fingerprints’ biometrics. Deployment also might be expensive as mostly the biometric fingerprint attendance system relies on hardware and peripherals, in addition to that, since biometric fingerprint will be the attendance system, meaning it must be available to each of the rooms where attendance is needed plus it is not ideal to remote settings.

According to (Truin, 2024), there is another one that is more reliable and has a higher accuracy than the fingerprint biometric attendance system and that is facial recognition. According to (Yang \& Han 2020), with the use of real time video processing, it can result in a high accuracy for about 82\% which is higher compared to other attendance systems. It can also reduce the truancy rates in school as the facial recognition system can easily identify who gets in and out in real time, preventing the students from cutting classes or even skipping classes.



\begin{comment}

%
% IPR acknowledgement: the contents withis this comment are from Ethel Ong's slides on RRL.
%
Guide on Writing your RRL chapter

1. Identify the keywords with respect to your research
      One keyword = One document section
                Examples: 2.1 Story Generation Systems
			 2.2 Knowledge Representation

2.  Find references using these keywords

3.  For each of the references that you find,
        Check: Is it relevant to your research?
        Use their references to find more relevant works.

4. Identify a set of criteria for comparison.
       It will serve as a guide to help you focus on what to look for

5. Write a summary focusing on -
       What: A short description of the work
       How: A summary of the approach it utilized
       Findings: If applicable, provide the results
        Why: Relevance to your work

6. At the end of each section,  show a Table of Comparison of the related works
   and your proposed project/system

\end{comment}

\section{Theme 1 Title}
This chapter  contains a review of research papers that:
%
% IPR acknowledgement: the following list of items are from Ethel Ong's slides RRL.
%
\begin{itemize}
\item Describes work on a research area that is similar or relevant to yours
\item Describes work on a domain that is similar or relevant to yours
\item Uses an algorithm that may be useful to your work
\item Uses a software / tool that may be useful to your work
\end{itemize}

%\section{Review of Related Software}
It also contains a review of software systems that:
%
% IPR acknowledgement: the following list of items are from Ethel Ong's slides on RRL.
%
\begin{itemize}
   \item Belongs to a research area similar to yours
   \item Addresses a need or domain similar to yours
   \item Is your predecessor
\end{itemize}

\section{Theme 2 Title}

\section{Chapter Summary}
Should include a table of related studies comparing them based on several criteria.

Highlight research gaps and the research problem.
