%   Filename    : chapter_1.tex 
\chapter{Introduction}
\label{sec:researchdesc}    %labels help you reference sections of your document

\section{Overview}
\label{sec:overview}

Attendance plays an important role in improving academic performance of students. There is evidence that students who has lower attendance often has lower grades\cite{EJES3887}. That's why it is usually enforced and recorded for most institutions. However, the systems in place for recording are usually manual and time consuming.

The traditional pen and paper attendance system has existed since the invention of paper itself. It is used for time keeping by manually writing or checking the 'present' status in a paper log book. Manually writing names takes an average of 17 seconds per student \cite{shoewu:2014}, and for class size of 30 students that leads to approximately 8 minutes wasted. While it is recognized that such system is time-consuming and wastes resources, it persisted because of it's familiarity. Going to class means bringing pen and paper for most students and teachers alike, so using the same material for recording attendance seemed the most practical.

In recent years, as laptops and portable computers became more accessible, some faculty of UP started transitioning to digital spreadsheets provided by services like Microsoft Excel. While it seemed to have moved the traditional pen and paper towards digitalization, another problem arises as this required manually roll calling students to say 'present'. It had the same problem of being a manual process. It is easily disrupted by a noisy class. Some time that was supposed to be utilized for immediate teaching was used for roll call.

Both systems mentioned are prone to errors and unnecessarily increases administrative burden for the faculty. Reduction in teaching time means frequently moving the lesson discussions by the faculty, with some topics being rushed or skip entirely by the end of semester. This reduces overall the quality of education students received and may negatively impact their readiness for subsequent courses they may take. 

Therefore, we propose a fully automatic, digital attendance system that addresses these concerns. We utilize the already distributed UP ID and pretrained face recognition models that ensures an easy, accurate, attendance keeping. It aims to ease the burden of faculty and students from manual methods of attendace system, allowing them to focus on class discussions instead.

\section{Problem Statement}
\textcolor{red}{DO NOT FORGET to write the statement of the research problem here, i.e., before the Research Objectives.}

A problem statement is your research problem written explicitly.  
The problem statement should do four things: 
\begin{enumerate}
	\item Specify and describe the problem (with appropriate citations) 
	\item  Provide evidence of the problem’s existence  
	\item Explain the consequences of NOT solving the problem  
	\item 	Identify what is not known about the problem that should be known.
	\item Subdivide the main problem into several subproblems.
\end{enumerate}


\section{Research Objectives}
\label{sec:researchobjectives}

\subsection{General Objective}
\label{sec:generalobjective}

This subsection states the over--all goal that must be achieved to answer the problem.
Address the following: Given your research challenge or opportunity, how do you intend  to solve it? What is the output of your research?


\subsection{Specific Objectives}
\label{sec:specificobjectives}

%
%  \begin{comment} ... \end{comment} is used for multiple lines of comment
%

This subsection is an elaboration of the general objective.  
It states the specific steps that must be undertaken to accomplish the general objective.  
These objectives must be \textbf{S}pecific, \textbf{M}easurable, \textbf{A}ttainable, \textbf{R}ealistic, \textbf{T}ime-bounded. Also, they are manageable and communicable.  

A specific objective start with ``to $<$verb$>$'' for example: to design/survey/review/analyze.

Studying a particular programming language or development tool (e.g., to study Windows/Object-Oriented/Graphics/C++ programming) to  accomplish the general objective is inherent in all thesis and, therefore, must not be included here.


\begin{comment}
% IPR acknowledgement: the following sentences and examples are from Ethel Ong's slides 
%     on Research Objectives
How to formulate your research objectives:
1. Identify what research steps do you need to perform to achieve your general objective.
2. Identify the questions that must be answered for you to achieve your general objective.
    Thereafter, convert these questions into action statements

Example #1:

Research Question:
  What are the general features of a web-based learning environment?

Specific Objective:
   To review existing web-based learning environment that teaches language learning for children


Example #2:

Research Question:
   How will you represent commonsense knowledge for use by computer systems?

Specific Objective:
   To identify knowledge representation approaches used by existing story generation systems

Example #3:
Research Question:
   What types of storytelling knowledge are needed to generate stories?

Specific Objective:
    To identify the different types of storytelling knowledge used in generating stories

Example #4:
Research Question:
    What machine learning approaches will you utilize?

Specific Objective:
    To determine existing machine learning algorithms [that can be used in training the computer system to detect cyberbullying cases] 

Example #5: Research Question:
    How will your research output be evaluated?

Specific Objective:
    To define evaluation metrics for validating the accuracy of the translation

\end{comment}

%
%  The following are example specific objectives; replace them with your own 
%

\begin{enumerate}
   \item To compare and contrast existing algorithms (on what problem?);
   \item To develop a new algorithm (for what purpose?)
   \item To analyze the algorithm (based on what criteria?)
\end{enumerate}


\section{Scope and Limitations of the Research}
\label{sec:scopelimitations}

This section discusses the boundaries (with respect to the objectives) of the research and the constraints within 
which the research will be developed.

\begin{comment}

%
% IPR acknowledgement: the sentences inside this comment are from Ethel Ong's slides on Scope and Limitations of the Research
%
Generally, one paragraph should be allotted for each of your research objectives.

Each paragraph contains a brief overview of the concept/theory and the purpose of doing the associated objective.

Each paragraph also includes a description of the scope/limitation of your study.

* Please refer to the slides for examples.

\end{comment}


\section{Significance of the Research}
\label{sec:significance}

This section explains why research must be done in this area.
 It rationalizes the objective of the research with that of the stated problem. 
 Avoid including sentences such as ``This research will be beneficial to the proponent/department/college'' as this is already an inherent requirement of all BSCS majors.  Focus on the research's contribution to the Computer Science field.

The following are guide questions that may help your formulate the significance of your research. 


%
% IPR acknowledgement: the following list of items are from Ethel Ong's slides on Significance of the Research
%
\begin{itemize}
\item  What is the relevance of your work to the computer science community? 

\begin{itemize} 
\item What will be your technical contributions, in terms of algorithms, or approaches, or new domain? 
\item What is your value-added compared to existing systems? 
\end{itemize}

\item What will be your contributions to society in general? 
    \begin{itemize}
      \item Who will benefit from your system? 
      \item Who are your target users and how will this system benefit them? 
   \end{itemize}
\end{itemize}

\begin{comment}
If applicable, describe possible commercialization and/or innovation in your research.
\end{comment}


