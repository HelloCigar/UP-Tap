%   Filename    : abstract.tex 
\begin{abstract}
The UP System started deployment of RFID/NFC-enabled UP ID in 2019. 5 years later, we have yet to see a system that fully utilizes the technology embedded in the UP ID. In particular, we see a great potential in using it as an access key for tracking the attendance of students in their classes. Professors currently either use the traditional pen and paper or a spreadsheet in their laptops to check for attendance. The mentioned practices are prone to forgery and takes precious time away from the class period. 

Our paper proposes a fully digital attendance tracking system that can be used by professors to record the attendance of their students in real time. The system uses UP ID and facial recognition for a two-layer validation process ensuring accuracy of the records. Facial recognition uses a pretrained Facenet model that surpasses human beings in multiple facial recognition tests for accuracy. The proposed system allows the students to check in by aligning their face in the camera, and tapping their ID to the RFID/NFC reader. The current prototype takes only about 2-3 seconds per student to complete the whole validation and recording process, with more room for optimizations down the line.




%From 150 to 200 words of short, direct and complete sentences, the abstract 
%should be informative enough to serve as a substitute for reading the entire SP document 
%itself.  It states the rationale and the objectives of the research.  
%In the final Special Problem  document (i.e., the document you'll submit for your final defense), the  %abstract should also contain a description of your research results, findings,  and contribution(s).

%  Do not put citations or quotes in the abract.

%Suggested keywords based on ACM Computing Classification system can be found at %\url{https://dl.acm.org/ccs/ccs_flat.cfm}

\begin{flushleft}
\begin{tabular}{lp{4.25in}}
\hspace{-0.5em}\textbf{Keywords:}\hspace{0.25em} & UP System, RFID, attendance, machine learning, facial recognition, Facenet model.\\
\end{tabular}
\end{flushleft}
\end{abstract}
