\chapter{Summary, Conclusions, and Recommendations}

\section{Summary}

This study proposed the integration of the UP RFID and facial recognition technology for an automated, secure, and efficient attendance system at the University of the Philippines Visayas. The system was designed to reduce the time and effort required by faculty and students to manage traditional manual and digital attendance methods. By utilizing a pretrained facial recognition model in combination with RFID, we aimed to provide a reliable and secure solution to track attendance in face-to-face classes.

Specifically, the system integrated the UP RFID's exclusive identification with real-time facial identification using the pretrained YOLOv8 and MobileFaceNet models with 1-2 second processing times per student. Major features were a Django-Nuxt web interface, hardware integration with Raspberry Pi, and two-factor authentication to maintain the CIA Triad principles (confidentiality, integrity, availability). Testing validated its superiority to manual roll calls and spreadsheet monitoring in terms of speed, accuracy, and fraud prevention, although light conditions and RFID reliance presented limitations.

The system's performance was evaluated using the Labeled Faces in the Wild (LFW) Dataset, which yielded an Area Under the Curve (AUC) score of 0.77. This indicates that there is a 77\% chance the model will correctly identify a genuine pair of faces over an impostor pair when randomly selected. 

\section{Conclusion}

The proposed system meets its objectives of improving the efficiency, security, and accuracy of attendance tracking. The integration of UP RFID and facial recognition technology addresses the gaps found in traditional methods, and its scalability suggests that it could be adopted by other universities and even large organizations in the future. Although the result is promising, further optimizations could improve performance, especially considering the constraints of the hardware used, such as the Raspberry Pi 5, which was limited to compressed models for faster recognition.

\section{Recommendation}

For future development and improvement of the system, the following recommendation are proposed:

\begin{enumerate}
	\item \textbf{Expand Applications Beyond Classroom Attendance }
	
	Expand the use case of the system such as in the library, time in time out of the student for their access logs. Additionally, the system is recommended to extend the scope of the system to the faculty as their attendance tracker, for examinations such as licensure or scholarships, the application of the syetem will be useful to authenticate the students efficiently before they take the exam.

	
	\item \textbf{Integration with Computerized Registration and Student Information System (CRS)}
	
	For the efficiency of the faculty and students, it is recommended to connect and integrate the system into CRS with the use of API or other data exchange formats that are supported by the CRS. This will automate the syncing of subject schedules of the students assigned to them.
	
	\item \textbf{Unified Application Packaging}
	
	Package the system as a standalone application using an installer like docker container, to simplify the installation, and include step-by-step documentation that guides the user for the initial setup.
	
\end{enumerate}
